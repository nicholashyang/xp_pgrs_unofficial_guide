\section{Research Symposium}

\subsection{XPGRS是什么}
XJTLU Postgraduate Research Symposium,简称XPGRS,或者Symposium,官方中文名是“博士生论坛”,是研究生院组织的活动。每年12月,会让全校的博士生一起用poster或者oral presentation展示自己的研究成果或者进展。每年的XPGRS,会在国庆节前后用邮件通知大家。

\subsection{我需要参加哪一项?}

官方要求是这样的:

\begin{itemize}
    \item 二年级的博士生需要做海报演讲
    \item 三年级的博士生需要做口头报告
    \item 四年级或以上的博士生将被邀请担任会议主席
    \item 也欢迎处于毕业论文阶段的硕士生参加
\end{itemize}

结合往年的实际情况,翻译过来就是:

\begin{itemize}
    \item 一年级的没事,但可以考虑做志愿者(会发邮件招募),或者直接去现场围观,了解前辈的工作
    \item 二年级的,必须做poster
    \item 三年级的,必须做oral presentation,可以考虑做chair
    \item 四年或以上就随便了
\end{itemize}

判断自己是哪一年级的,以当年12月1日为准。这对3-9月入学的不难,但有很多同学就恰好是在12月1日入学的。据22年的消息,这部分同学是需要做的,也就是比如21年12月入学,22年需要做poster。
% 这部分同学在一年后可以自己选择做不做。例如
% \begin{itemize}
%     \item 奥观海同学,20年12月入学。在21年12月,他发现自己做的东西太少,实在是展示不出来,于是决定不参加。这样,他在22年12月必须做poster,23年12月必须做oral presentation,24年就没事了。
%     \item 川建国同学,20年12月入学。21年12月她觉得还算可以展示一下,以及想尽早锻炼下能力,也算是为后面一年可能参加的学术会议做准备。于是她21年12月参加了poster,22年12月则必须做oral presentation,23年就没事了。
% \end{itemize}
%
当然,由于12月入学同学确实特殊,政策可能会变,因此最好每年发邮件(或者打电话)问研究生院。

理论上,其实每年你都可以同时参加poster和oral,只要你愿意!甚至,也可以从来都不去!但需要导师同意。一般没有特殊原因,导师也是希望你去锻炼下的。如果实在有原因,需要导师和研究生院联系,同意了就可以不去。(我听说过有的导师觉得symposium没卵用直接让学生从来都不参加,也有的导师,第一次就让学生同时做poster+oral...)

\begin{figure}[H]
    \caption{2021 Symposium 的 poster 会场(CB G13W),正式开始前夕}
    \centering
    \includegraphics[width=\columnwidth]{author-folder/Kai.Wu/synposium_poster.jpg}
\end{figure}

\subsection{有什么用}

\begin{enumerate}
    \item 锻炼作报告的能力,锻炼英语口语,获得校内其他老师的指导
    \item 可以得奖。学校会请几位老师来听你讲海报、讲PPT,并给你打分。会选出最佳海报奖(10\%)最佳演讲奖(10\%)优秀海报奖(20\%)优秀演讲奖(20\%),会有一张奖状,外加1000和500元的……会议经费(会议经费使用参见章节\ref{sec.fund}),虽然不是现金奖励,但还算有用吧\sout{(抠门学校)}
\end{enumerate}

\begin{figure}[H]
    \centering
    \includegraphics[width=0.4\columnwidth]{author-folder/Kai.Wu/poster_award.jpg}
\end{figure}

\subsection{“我不想去”——尽量不要!}

经常听到有同学这么想:奖励这么少,含金量又低,即使最后拿个最佳也没卵用。而且即使不去,最后也没能把我怎么样。是的,坦白了讲,一来,导师同意,就可以合法不去;二来,即使直接装消失,最后也不会怎么样,和毕业无关。

但,作为参加过poster、oral,也chair过三场oral、听过很多口头报告的老人,除非你作报告真的可以信手拈来,我真的强烈建议你们去参加!我见过有同学全程低头念稿,有同学直接把稿子写PPT上读,有同学过于紧张手和声音都抖个不停,有同学PPT分不清楚主次、重点不突出,有同学面对老师的问题手足无措。这些任何一项,放到你毕业答辩,都是致命的!!须知,咱答辩不是和体制内大学一样走个过场,是真枪实弹的必须要做好报告、再回答好每一个尖锐的问题。\textbf{你想要把这些问题直接暴露在答辩上,还是提前发现,多加锻炼,早些解决?}

所有,既然在Symposium里,你不会Fail,只会评奖,评委老师也都很友善,会帮你找出重要问题。所以,\uline{\textbf{是非常安全的、绝佳的【锻炼机会】}}。可不能等到答辩了或者等到你参加顶会了再去锻炼啊!

\subsection{注意事项、如何表现出色}

下面附上我在2022年symposium看到的口头报告打分表(可放大查看),poster的注意事项其实很类似(参加poster的,也需要准备一个3分钟左右的talk来给到场的评委老师介绍)。坦白的说,这些项目放在任何报告里都是值得注意的地方,可以用来仔细审视自己的海报。

\begin{figure}[H]
    \caption{2022年的打分表}
    \centering
    \begin{tabular}{rl}
        \includegraphics[width=0.5\columnwidth]{author-folder/Kai.Wu/2022_XPGRS_oral_judging_criteria.pdf} & 
        \includegraphics[width=0.5\columnwidth]{author-folder/Kai.Wu/2022_XPGRS_oral_judging_form.pdf} 
    \end{tabular}
\end{figure}

如果你想表现出色、得奖,可以继续看下面的内容:

作报告和做海报,大家在网上都能搜到大量教程。但,XPGRS和其他学术报告最大的区别就是,绝大多数同学、包括学校的评审老师,非常有可能完全不知道你的研究领域(正所谓,隔行如隔山)。如果你按照跟你导师汇报的方式来讲,或者是把你在某次内行齐聚的学术会议上的海报/PPT原封不动拿来用,有可能是得不了奖的,因为,评委老师和其他同学确实听不懂。

我的导师告诉我,XPGRS因为是给外行讲,和学术报告其实不同,更偏向科普性质。如何在短时间内,让一个外行对你的研究感兴趣,并且搞懂个70\%,也是Symposium的部分目的。如果你不知道该怎么做,可以想象有一天你在电梯里遇到了席酉民,怎么在短时间内让他知道你在做什么,又有什么用,最好还能让他感兴趣。或者是,过年回家跟你哪个亲戚或者老同学怎么讲清楚你在做什么。学术不是闭门造车,如何把自己的学术讲出去,让学术成果造福其他人也是很重要的技能。

\begin{flushright}
(2022年10月19日 by \Wu)

(major update: 2022年12月16日 by \Wu)
\end{flushright}