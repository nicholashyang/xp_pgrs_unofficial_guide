\chapter{写作指南}
\label{chapter.author-ins}

% v0.1 by Kai.Wu

你用十几分钟写下的经验,可能会节约下后来人几小时甚至几天的时间。欢迎为本攻略做出贡献。

最简单粗暴的贡献方式,直接把纯文本、word或者tex文件发给 \Shiyao。管理员会帮你把文件转换成\LaTeX{}格式放进攻略。虽然管理员可能没有那么开心,因为要费些时间建文件整理上传,但也是可以的。如果你心疼可怜的管理员,可以看看下面的方法(嫌麻烦的同学可以跳过下面的部分,直接看本页最后一段话)。更好的投稿姿势:


\begin{itemize}
    \item 如果你有使用Git和交Pull Request的经验,也会写\LaTeX{},那么可以直接在\href{https://github.com/xp-pgrs-unofficial-guide/xp_pgrs_unofficial_guide}{本手册的GitHub页面}提交PR。\href{https://www.zhihu.com/question/21682976/answer/79489643}{如何交PR可参照这个链接}
    % \item 如果你不会交PR,但会写\LaTeX{},可以在Overleaf上写。请发邮件到 Kai.Wu19 at student.xjtlu.edu.cn 申请编辑权限。写完过后,请发邮件给管理员把稿件推送到GitHub。
\end{itemize} 

\vspace{5mm}
注意:对于上面高级的投稿方法,为避免多人协作弄乱顺序,请每位作者在\texttt{author-folder}文件夹下建立自己的子目录,在里面新建\texttt{tex}文件写作,第一行直接从\texttt{section}命令开始,其后写正文。写完了过后,在正确的章节里,用以下代码把你的稿件插入到\texttt{chapter}文件夹下的章节里
\begin{lstlisting}
    \input{你的tex文件路径}
\end{lstlisting} 
例如,王多鱼同学想要分享自己的常用软件。首先在\texttt{author-folder}下建立一个\texttt{duoyu.wang}文件夹,在里面新建\texttt{health-insurance-use.tex},第一行写\lstinline[breaklines=true]!\section{学生医保使用}!写标题,后面写正文。写好过后,在\texttt{chapters}文件夹里的\texttt{others.tex}中,添加一行\lstinline[breaklines=true]!\input{author-folder/duoyu.wang/health-insurance-use.tex}!,就完成了。

\vspace{5mm}
如果要一些高级的编辑方式,可参照项目 GitHub 的\texttt{author-guide}目录下的模板手册和例子。

\vspace{5mm}
属不署名都可以。可以匿名,也可以把姓名邮箱院系入学年份都写上。你如果愿意的话把生辰八字、出生年月、房产车产详情附上也是完全可以的,这样甚至可以问问PGR Society能不能安排相亲(雾)。最后必须要提醒:您不得发布任何违反中华人民共和国法律、西交利物浦大学博士生守则、\href{https://www.liverpool.ac.uk/aqsd/academic-codes-of-practice/pgr-code-of-practice/}{利物浦大学博士生守则}和其他任何适用规定的内容。您需要对你发布的内容负责,PGR Society无法对内容的准确性做担保或审核。由您发布的内容导致的任何纠纷,PGR Society社团和其他作者不承担任何连带责任。

\vspace{5mm}
再次感谢你为学弟学妹们、祖国的花朵们做出的贡献  (ง •̀\_•́)ง